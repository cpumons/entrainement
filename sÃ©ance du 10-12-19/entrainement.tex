\documentclass[a4paper]{article}

\usepackage{tabularx}
\usepackage{graphicx}
\usepackage{url}
\usepackage[utf8]{inputenc}
\usepackage[T1]{fontenc}
\usepackage[french]{babel}
\usepackage{amsmath, amssymb, amsthm, verbatim}
\usepackage{IEEEtrantools}
\usepackage[margin=1in]{geometry}
\usepackage[colorlinks, linkcolor=blue]{hyperref}
\usepackage{epigraph}
\usepackage{mathrsfs}
\usepackage[toc,page]{appendix}
\usepackage{tikz-cd}
\usepackage{xcolor}
%Header stuff
%\usepackage{fancyhdr}
%\pagestyle{fancy}
%\renewcommand{\chaptermark}[1]{\markboth{#1}{}}
%\renewcommand{\sectionmark}[1]{\markright{\thesection\ #1}}
%\fancyhf{}
%\fancyhead[LE,RO]{\bfseries\thepage}
%\fancyhead[LO]{\bfseries\rightmark}
%\fancyhead[RE]{\bfseries\leftmark}
%\renewcommand{\headrulewidth}{0.5pt}
%\renewcommand{\footrulewidth}{0pt}
%\setlength{\headheight}{14.85pt} % space for the rule
%\fancypagestyle{plain}{%
%  \fancyhead{} % get rid of headers on plain pages
%  \renewcommand{\headrulewidth}{0pt} % and the line
%}

\title{Séance d'exercice 1}
\author{CPUMons}
\date{\today}


\begin{document}

\maketitle

\paragraph{Avant de commencer...}
Ce document comporte plusieurs exercices dont la difficulté est renseignée par un nombre et est \emph{globalement} croissante. Il est évidemment fortement recommandé de réfléchir au préalable sur papier avant de se lancer dans la partie implémentation.\\
Répartition de la difficulté:
\begin{itemize}
\item Niveau 1: problème facile, ne nécessite pas de code complexe pour être mis en œuvre;
\item Niveau 2: problème moyen, demande une réflexion plus poussée sur la présentation du code;
\item Niveau 3: problème délicat, demande une réflexion poussée sur le problème en soi.
\item Niveau 4: problème difficile.
\end{itemize}
\begin{comment}
\section{Rappel (ou non): lire une entrée}

\subsection{En théorie:}
En Python3, la fonction \emph{input()} est utilisée pour lire une entrée via l'invite de commande.\\
\emph{Entrée:} Aucun argument n'est obligatoire, mais on peut aussi fournir une chaîne de caractère qui sera affichée afin que l'utilisateur sache ce qu'il doit entrer.\\
\emph{Sortie:} L'entrée fournie par l'utilisateur, sous forme de chaine de caractère (\emph{String}).\\
Il est donc important de convertir la donnée retournée!
\subsection{En pratique:}
Tâche: additionner deux nombres donnés en entrée et afficher le résultat.\\
\begin{verbatim}
nbr1 = input("Votre premier nombre : ")
nbr2 = input("Votre second nombre : ")
res = int(nbr1) + int(nbr2)
print(res)
\end{verbatim}
\end{comment}


\section{Aidez CPUMons !! (niveau 2)}
Le Kot associatif CPUMons a besoin d'aide ! 

La liste des membres commence à s'allonger et il serait judicieux d'établir un système de gestion de membre avant que cette liste devienne ingérable.
Pour cela nous avons besoin:
\begin{itemize}
\item D'un objet membre qui a comme caractéristiques: nom, prénom, adresse email umons, faculté, section, date d'inscription (sous la forme d'un nombre aammjj), son nombre de participation, ainsi que les dates auxquelles il a participé à un entraînement (aammjj).
\item L'adresse email doit être construite sur base du prénom et du nom (prénom.nom@student.umons.ac.be).
\item On a besoin de connaître quels membres étaient présents à une date donnée ainsi que le nombre de membres présents à cette séance.
\item Pour établir un graphique du nombre de membres présents à chaque date, on a besoin de connaître le nombre d'étudiants pour toutes les séances.
\item Les membres seront stockés dans un tableau.
\item On doit pouvoir ajouter, supprimer des membres du tableau.
\end{itemize}

\section{classement en moyenne (niveau 1.5)}
Le but de cet exercice est de diviser une liste en deux sous-listes par rapport à la moyenne des élements de la liste de base
\paragraph{exemple : }
Soit L une liste : {2,4,95,7,32,-4,66,26,13} on peut calculer la moyenne qui est égale à 26.78 et donc les deux listes en sorties seront L1 : {2,4,7,-4,26,13} et L2 : {95,32,66}
\paragraph{remarque : }
\begin{itemize}
\item la complexité doit être en maximum O($n^2$)
\item Vous ne pouvez créer qu'une nouvelle liste et non deux.
\item Si un élement est égal à la moyenne il doit apparaitre dans les deux listes
\end{itemize}



\section{...}





\section{...}





\section{...}



\section{...}


\section{Concours}
Pour ceux qui veulent s'entraîner pour participer aux concours, voici quelques liens pour vous permettre de travailler sur ce qui vous intéresse à votre propre rythme:\\
\begin{enumerate}
\item \href{http://www.france-ioi.org/algo/chapters.php}{FranceIOI}: nécessite de débloquer les premiers niveaux, mais recouvre une très grande variété d'algorithmes et de problèmes différents rangés par thématique;
\item \href{https://www.isograd.com/FR/solutionconcours.php?contest_id=46&que_str_id=&reg_typ_id=2&fbclid=IwAR3A3eAVzJVquecv1_MaLoWCMhLxku8SxbQl_awOcJcDevuJQ2HZtNHmupg}{Isograd}: sur ce site, vous retrouverez de nombreux anciens concours dont, en particulier, les BattleDev précédentes qui constituent un \emph{excellent} point de départ dans le monde des concours;
\item \href{https://codingcompetitions.withgoogle.com/codejam?fbclid=IwAR1gU6uObu-gYzAOAdsR4uvmBvfhImyiSUi1pMxEEXkPVFA6MTnWVZ1etT0}{Google Code Jam}: les énoncés des années précédentes y sont disponibles. Le niveau requis est, bien évidemment, progressif et les problèmes sont en général assez intéressants;
\item \href{https://open.kattis.com/}{Kattis}: site reprenant de \emph{très} nombreux problèmes. Il vous est conseillé de tester un ou deux problèmes "triviaux" afin de bien vérifier si vous n'avez pas de problèmes avec les entrées et sorties ; puis, de passer aux faciles et, rapidement, aux moyens (les difficiles portant très bien leur nom).
\end{enumerate}

\end{document}











