\documentclass[a4paper]{article}

\usepackage{tabularx}
\usepackage{graphicx}
\usepackage{url}
\usepackage[utf8]{inputenc}
\usepackage[T1]{fontenc}
\usepackage[french]{babel}
\usepackage{amsmath, amssymb, amsthm, verbatim}
\usepackage{IEEEtrantools}
\usepackage[margin=1in]{geometry}
\usepackage[colorlinks, linkcolor=blue]{hyperref}
\usepackage{epigraph}
\usepackage{mathrsfs}
\usepackage[toc,page]{appendix}
\usepackage{tikz-cd}
\usepackage{xcolor}
%Header stuff
%\usepackage{fancyhdr}
%\pagestyle{fancy}
%\renewcommand{\chaptermark}[1]{\markboth{#1}{}}
%\renewcommand{\sectionmark}[1]{\markright{\thesection\ #1}}
%\fancyhf{}
%\fancyhead[LE,RO]{\bfseries\thepage}
%\fancyhead[LO]{\bfseries\rightmark}
%\fancyhead[RE]{\bfseries\leftmark}
%\renewcommand{\headrulewidth}{0.5pt}
%\renewcommand{\footrulewidth}{0pt}
%\setlength{\headheight}{14.85pt} % space for the rule
%\fancypagestyle{plain}{%
%  \fancyhead{} % get rid of headers on plain pages
%  \renewcommand{\headrulewidth}{0pt} % and the line
%}

\title{Séance d'exercice 3}
\author{CPUMons}
\date{\today}


\begin{document}

\maketitle

\paragraph{Avant de commencer...}
Ce document comporte plusieurs exercices dont la difficulté est renseignée par un nombre et est \emph{globalement} croissante. Il est évidemment fortement recommandé de réfléchir au préalable sur papier avant de se lancer dans la partie implémentation.\\
Répartition de la difficulté:
\begin{itemize}
\item Niveau 1: problème facile, ne nécessite pas de code complexe pour être mis en œuvre;
\item Niveau 2: problème moyen, demande une réflexion plus poussée sur la présentation du code;
\item Niveau 3: problème délicat, demande une réflexion poussée sur le problème en soi.
\item Niveau 4: problème difficile.
\end{itemize}
\begin{comment}
\section{Rappel (ou non): lire une entrée}

\subsection{En théorie:}
En Python3, la fonction \emph{input()} est utilisée pour lire une entrée via l'invite de commande.\\
\emph{Entrée:} Aucun argument n'est obligatoire, mais on peut aussi fournir une chaîne de caractère qui sera affichée afin que l'utilisateur sache ce qu'il doit entrer.\\
\emph{Sortie:} L'entrée fournie par l'utilisateur, sous forme de chaine de caractère (\emph{String}).\\
Il est donc important de convertir la donnée retournée!
\subsection{En pratique:}
Tâche: additionner deux nombres donnés en entrée et afficher le résultat.\\
\begin{verbatim}
nbr1 = input("Votre premier nombre : ")
nbr2 = input("Votre second nombre : ")
res = int(nbr1) + int(nbr2)
print(res)
\end{verbatim}
\end{comment}


\section{Aidez CPUMons !!! (niveau 2)}
Le Kot associatif CPUMons a besoin de votre aide! 

La liste des membres commence à s'allonger et il serait judicieux d'établir un système de gestion des membres avant que cette liste ne soit trop grande.
Pour cela nous avons besoin:
\begin{itemize}
\item D'un objet "Membre" qui se caractérise par: un nom, un prénom, une adresse email UMons, une faculté, une section, une date d'inscription (sous la forme d'un nombre aammjj), son nombre de participation, ainsi que les dates auxquelles il a participé à un entraînement (aammjj).
\item L'adresse email doit être construite sur base du prénom et du nom de l'étudiant (prénom.nom@student.umons.ac.be).
\item On a besoin de connaître quels membres étaient présents à une date donnée ainsi que le nombre de membres présents à cette séance.
\item Pour établir un graphique du nombre de membres présents à chaque date, on a besoin de connaître le nombre d'étudiants pour toutes les séances.
\item Les membres seront stockés dans un tableau.
\item On doit pouvoir ajouter et supprimer des membres du tableau.
\end{itemize}

\section{Classement en moyenne (niveau 1.5)}
Le but de cet exercice est de diviser une liste en deux sous-listes par rapport à la moyenne des élements de la liste de base.
\paragraph{Exemple: }
Soit $L$ une liste: {2,4,95,7,32,-4,66,26,13}, on peut en calculer la moyenne qui est égale à 26.78 et donc, les deux listes en sorties seront $L_1$: {2,4,7,-4,26,13} et $L_2$ : {95,32,66}.
\paragraph{Remarques/Restrictions: }
\begin{itemize}
\item La complexité doit être, au maximum, en $O(n^2)$;
\item Vous ne pouvez créer qu'une nouvelle liste et non deux;
\item Si un élément est égal à la moyenne, il doit apparaitre dans les deux listes.
\end{itemize}
\paragraph{Challenge (niveau 2.75): }
Votre algorithme doit être en $O(n \log (n))$, mais il vous est autorisé d'utiliser les instructions du type t$[2:5]$ (où t est une liste).


\section{Critique de l'exploitation animale dans les champs de salsifis. (niveau 2.5)}

Votre aide est requise par la Corporation des Producteurs et Utilisateurs Mexicains d'Ornithorynques pour Nourrir des Salsifis. En effet, cette noble entreprise étudie l'efficacité des ornithorynques dans l'entretient des champs de salsifis au Mexique. La capacité de cet animal pour entretenir les champs de salsifis est, il faut l'avouer, fortement sous-estimée. De nombreuses études ont montré que des ornithorynques, bien positionnées dans un champ de salsifis, entretenaient parfaitement tous les salsifis proches d'eux.

Vous avez été embauché par la C.P.U.M.O.N.S. pour leur fournir un algorithme leur permettant d'évaluer le nombre de plans de salsifis dont s'occupe un ensemble donné d'ornithorynques.

Le champ est représenté par une grille carrée dont on vous donne la longueur du coté, $N$. Sur cette grille sont placés $M$ ornithorynques ($0 \leqslant M < N$). Sachant qu'un ornithorynque s'occupe de tous les plans de salsifis sur les 8 cases adjacentes à sa position, vous devez calculer le nombre de plans de salsifis qui peuvent être entretenus au maximum. Il est à noter que, le salsifis étant une plante très fragile, si plus d'un ornithorynque s'occupe d'un plan donné, celui-ci sera trop nourri et mourra malheureusement. (A noter, bien entendu, qu'une case contenant un ornithorynque ne peut pas contenir de plan de salsifis.)

\paragraph{Entrée :}
\begin{itemize}
\item Sur la première ligne, se trouvent les deux entiers $N$ et $M$;
\item Les $M$ lignes suivantes contiennent chacune deux entiers représentant la position des $M$ ornithorynques (on suppose toutes les positions distinctes).
\end{itemize}
\paragraph{Sortie :}
\begin{itemize}
\item Le nombre maximal de plans de salsifis qui peuvent être entretenus par les ornithorynques.
\end{itemize}

\paragraph{Remarque :}
Afin de rendre le problème un peu plus court, on suppose que les positions des ornithorynques sont toutes telles que les 8 cases adjacentes existent (donc qu'il n'y a pas d'ornithorynque en bord de champ).

\section{Une histoire déjà vue... (niveau 2)}
Dans une petite ville, comme il en existe tant d'autres, s'était implantée une gigantesque chocolaterie qui faisait vivre toute la région par son ampleur et son développement. En effet, la conjoncture économique courante avait été mise à mal par la pénultième crise financière survenue deux ans auparavant, mais qui n'avait été que doux euphémisme par rapport à celle de l'année suivante qui, bien qu'antithétique par son origine, avait fait de la ville le palliatif apanage de la misère; tel un abject homoncule\footnote{Ce mot étant la seule raison pour laquelle ce texte n'est pas en vers.} devant le méprisable, mais inexorable zéphyr de la vie... Voici donc le contexte, somme toute commun mais malheureusement compréhensible, dans lequel s'était implantée cette grande chocolaterie: la Chocolaterie Préférée de l'Union des Médiocres Ouvriers Nouvellement Sous-employés.\\
\par Un jour, le PDG annonça qu'un grand concours était lancé afin qu'un enfant puisse visiter la C.P.U.M.O.N.S.\footnote{Aucun lien de parenté avec votre kot-associatif préféré.}. Pour remporter le concours, il fallait trouver un ticket doré caché dans une des barres chocolatées de la société. Cependant, le PDG ajouta un indice: "Le ticket doré a été caché par mes soins dans une barre de chocolat, mais pas n'importe quelle barre! La somme récursive des chiffres de son numéro de série est la réponse à la vie, l'univers et tout le reste! Bonne chance pour le trouver!"
\par Intrigué par le terme de "somme récursive", vous apprenez que c'est un calcul assez simple à faire: après avoir fait la somme de tous les chiffres du numéro de série, si le nombre obtenu a strictement plus de 2 chiffres, vous faites la somme récursive du nombre obtenu. Jusqu'à obtenir un nombre avec, au plus, deux chiffres.
\par Vous décidez donc d'écrire un petit programme qui, étant donné le numéro de série d'une barre de chocolat, vous informe si cette barre a des chances d'être gagnante ou non...

\paragraph{Challenge (niveau 3):} Obtenir une complexité dans le pire des cas en $O(n)$ où $n$ est le nombre de chiffres du numéro de série.

\section{Limiter le hasard... D'un jeu de hasard! (niveau 4)}
Vous avez été engagé par la Compagnie des Probabilités pour l'Utilisation Modérée mais Onéreuse de Nos Sous afin d'étudier un jeu de hasard relativement simple et paillard: le "cul de chouette"\footnote{Nom tout aussi paillard.}. Ce jeu se joue avec un nombre $N$ de dés à 6 faces ($1 \leqslant N \leqslant 10$) que l'on lance une fois (les dés n'étant évidemment pas truqués. De plus, l'important, dans ce jeu, ce sont les valeurs: en l'occurrence la somme des résultats des $N$ dés.\\
Votre tâche est d'écrire un algorithme donnant le nombre de manières possibles d'obtenir un résultat $M$ ($N \leqslant M \leqslant 6N$). Pour ce faire, il vous est conseillé d'effectuer les calculs pour $N = 2 et 3$ afin de pouvoir généraliser la méthode.
\paragraph{Remarque:} au vu de la faible valeur de $N$, il n'est pas nécessaire de se soucier de la complexité de l'algorithme créé.

\section{Concours}
Pour ceux qui veulent s'entraîner pour participer aux concours, voici quelques liens pour vous permettre de travailler sur ce qui vous intéresse à votre propre rythme:\\
\begin{enumerate}
\item \href{http://www.france-ioi.org/algo/chapters.php}{FranceIOI}: nécessite de débloquer les premiers niveaux, mais recouvre une très grande variété d'algorithmes et de problèmes différents rangés par thématique;
\item \href{https://www.isograd.com/FR/solutionconcours.php?contest_id=46&que_str_id=&reg_typ_id=2&fbclid=IwAR3A3eAVzJVquecv1_MaLoWCMhLxku8SxbQl_awOcJcDevuJQ2HZtNHmupg}{Isograd}: sur ce site, vous retrouverez de nombreux anciens concours dont, en particulier, les BattleDev précédentes qui constituent un \emph{excellent} point de départ dans le monde des concours;
\item \href{https://codingcompetitions.withgoogle.com/codejam?fbclid=IwAR1gU6uObu-gYzAOAdsR4uvmBvfhImyiSUi1pMxEEXkPVFA6MTnWVZ1etT0}{Google Code Jam}: les énoncés des années précédentes y sont disponibles. Le niveau requis est, bien évidemment, progressif et les problèmes sont en général assez intéressants;
\item \href{https://open.kattis.com/}{Kattis}: site reprenant de \emph{très} nombreux problèmes. Il vous est conseillé de tester un ou deux problèmes "triviaux" afin de bien vérifier si vous n'avez pas de problèmes avec les entrées et sorties ; puis, de passer aux faciles et, rapidement, aux moyens (les difficiles portant très bien leur nom).
\end{enumerate}

\end{document}











